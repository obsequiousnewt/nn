%-----------------------------------------------------------------------------
%
%               Template for sigplanconf LaTeX Class
%
% Name:         sigplanconf-template.tex
%
% Purpose:      A template for sigplanconf.cls, which is a LaTeX 2e class
%               file for SIGPLAN conference proceedings.
%
% Guide:        Refer to "Author's Guide to the ACM SIGPLAN Class,"
%               sigplanconf-guide.pdf
%
% Author:       Paul C. Anagnostopoulos
%               Windfall Software
%               978 371-2316
%               paul@windfall.com
%
% Created:      15 February 2005
%
%-----------------------------------------------------------------------------


\documentclass[preprint]{sigplanconf}

% The following \documentclass options may be useful:

% preprint      Remove this option only once the paper is in final form.
% 10pt          To set in 10-point type instead of 9-point.
% 11pt          To set in 11-point type instead of 9-point.
% numbers       To obtain numeric citation style instead of author/year.

\usepackage{amsmath}

\newcommand{\cL}{{\cal L}}

\begin{document}

\special{papersize=8.5in,11in}
\setlength{\pdfpageheight}{\paperheight}
\setlength{\pdfpagewidth}{\paperwidth}

%\conferenceinfo{CONF 'yy}{Month d--d, 20yy, City, ST, Country}
%\copyrightyear{20yy}
%\copyrightdata{978-1-nnnn-nnnn-n/yy/mm}
%\copyrightdoi{nnnnnnn.nnnnnnn}

% Uncomment the publication rights you want to use.
%\publicationrights{transferred}
%\publicationrights{licensed}     % this is the default
%\publicationrights{author-pays}

\titlebanner{CSC 395: Modern Programming Principles}        % These are ignored unless
\preprintfooter{Final project for CSC 395}   % 'preprint' option specified.

\title{Solving NP-Hard Problems with Neural Networks}
\subtitle{Taking the ``Artificial'' out of Artificial Intelligence}

\authorinfo{Zebediah Figura}
           {Grinnell College}
           {figuraze@grinnell.edu}
           
\authorinfo{Theodoros Kalfas}
           {Grinnell College}
           {kalfasth@grinnell.edu}
           
\authorinfo{Daniel Nanetti-Palacios}
           {Grinnell College}
           {nanettip@grinnell.edu}


\maketitle

\begin{abstract}


\end{abstract}

\category{CR-number}{subcategory}{third-level}

\keywords
Neural Networks, Machine Learning, Haskell

\section{Introduction}

The concept of neural networks, while dating back to the 1940s, has gained popularity and interest in recent years, partly due to breakthroughs in fast GPU-based implementation and the invention of recurrent neural nets. In our paper we implement a neural net in Haskell, a strongly-typed functional language, with the intent of demonstrating how a neural network can solve NP-hard problems. In order to best demonstrate the process and successfulness of a neural network, we present as the problem to be solved the well-known video game \textit{Snake}. Solving a problem such as this allows us to observe elements of the network's heuristic, as well as show the speed and manner in which it learns to solve the problem.

A neural network functions by accepting a set of inputs, then feeding those inputs through a series of neurons. Each neuron receives some number of numerical inputs, either directly taken from the inputs to the network or from the output of some other neurons, and multiplies those inputs by an arbitrary set of weights, outputting the sum (i.e. the dot product) to another neuron. The outputs of the last set of neurons may then be interpreted. In the case of \textit{Snake} the set of inputs is simply the set of all cells, with (say) -1 representing the presence of a barrier, 0 an empty cell, and 1 representing a fruit. The program would then give four outputs, representing (approximately) the probabilities with which the player should turn left, right, up, and down, respectively.

The key function here which allows the program to actually solve the problem itself, rather than having all weights be programmed manually, is the concept of \textit{machine learning}. In order to find the most optimal configuration of weights which will allow the network to most effectively solve the problem, one might arbitrarily modify weights and test whether the network's output is more or less accurate. However, since there may be as many as billions of weights in the network, systematically finding the best combination is infeasible in this manner. A better solution involves using calculus to determine the partial derivative of the output's accuracy (judged by some \textit{cost function}) with respect to each neuron, and then changing all neurons according to the sign and magnitude of this derivative. The most efficient way of doing this, popularized in 1986 by Rumelhart et al., is the \textit{back-propagation} algorithm, makes use of the chain rule to calculate all modifications using only two ``passes'' through the network.

\section{Prior Work}
%%%% Please read and review this part, tell me if I should add something, or add it yourselves.
\t A tremendous amount of work has been conducted in the area of Game AI using neural networks. Ranging from the simplest games such as snake, to complex games such as Go with the recent victory of Google's AlphaGo program over renowned Go champion Lee Sedol. The game we will be focusing on, Snake, has definitely been solved using the simple approach of finding the snake's head, looking at the block that it's about to occupy in the next frame, and if it's already occupied by its body, make the snake turn, maybe with adjustments to choose the best way to turn. However, we are seeking to implement a solution using a neural network, so finding other relatively simple neural networks that are developed to solve games can be useful.\\
Several programs which beat arbitrary Mario-style levels have been made, including the open-source MarI/O by SethBling. A Mario AI Championship used to be conducted where programmers would submit code, so a lot of people have tried and succeeded in beating platformer levels in games. Other "AI" type neural network guides are available such as \textit{How to build Neural Networks for Games} by Penny Sweetser or \textit{Promising Game AI Techniques} by Steve Rabin.\\
As a general introduction to Neural Networks, Michael Nielsen's online textbook \textit{Neural Networks and Deep Learning} is an amazing resource that can give anyone a jump-start on all the research that has happened since the conception of neural networks. Core concepts that need to be understood are perceptrons, the equivalents to neurons, that take some weighed inputs and provide an output. If a lot of them are chained, they make a net comprised of different layers that depend on the inputs of the previous layer, thus they can make decisions depending on previous more simple decisions. Finally, the basis on which neural networks improve is changing the weights on each neuron so that the results are more accurate. It can be shown that small changes in the weights result in small changes in the output, so if during the training stages of the neural network, there is a small discrepancy between the expected result and the actual one, small changes in the weights can provide us with a better network. The actual changes can be done arbitrarily until we reach a consistent network, or they can be based on computations.\\
\t The general approach to building a neural network which plays a platformer game intelligently doesn't differ much from classic neural networks such as character recognition. The inputs are either screenshots of the game's execution or the memory of the game, both translated so that only the important patterns such as platforms, enemies and power-ups are present, and the outputs are button presses, which are then fed into the game. The approach that SethBling takes is to randomly generate the first layer of neurons, and let each generated neural network run on an emulation of Mario. The ones that perform the best are kept and built upon, introducing new layers of neurons on them, and tested as well. That is, the neural network ends up training itself, or rather, the group of neural networks that are generated end up choosing the best one, instead of looking at other trials or being fed data that will help them train. This is a much more complex game than Snake, so implementing a similar network for Snake will be feasible during the timeline of this project.

\section{Proposed Work}
%%%% Its not a lot but I don't how to flesh this out any more since I'm still a bit unsure of the exact things we need to do to get this done.
\t For this project we plan on producing a program that runs a game and makes the most optimal choices at each frame of gameplay. The program will use neural networks that will output the best move to make at the current state of the game in order to get closer to the goal. Within this project we will develop the neural network that will handle this task and learn by going through various different training data. This will be the biggest task of the project as we would need to create a neural network ourselves using Haskell that can learn through various trails. Other things we will need to develop is how the program will read each frame of the game and interpret that data so it can go through the neural network.

\section{Timeline}
%%%% Timeline feels a bit straightforward since there are only 3 checkpoints. If you guys know how to make this longer go ahead.
\t By the time we reach the first checkpoint, we hope to have at least completed a very basic structure for our neural network that we will use for the rest of the project. We also plan on figuring out how to get the data from the emulator that the game is running on and use it as the input for the program. For the second checkpoint, our neural network will hopefully be able to reach the goal in most if not all the tests given, but not exactly optimally. When the third checkpoint comes around, we would either have turned the neural network into recurrent neural network or have optimized and adjusted the previous neural network so the player would reach the goal in the most optimal way possible.

\appendix
\section{Appendix Title}

This is the text of the appendix, if you need one.

\acks

Acknowledgments, if needed.

% We recommend abbrvnat bibliography style.

\bibliographystyle{abbrvnat}

% The bibliography should be embedded for final submission.

\begin{thebibliography}{}
\softraggedright

\bibitem[Smith et~al.(2009)Smith, Jones]{smith02}
P. Q. Smith, and X. Y. Jones. ...reference text...

\end{thebibliography}


\end{document}
