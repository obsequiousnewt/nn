%-----------------------------------------------------------------------------
%
%               Template for sigplanconf LaTeX Class
%
% Name:         sigplanconf-template.tex
%
% Purpose:      A template for sigplanconf.cls, which is a LaTeX 2e class
%               file for SIGPLAN conference proceedings.
%
% Guide:        Refer to "Author's Guide to the ACM SIGPLAN Class,"
%               sigplanconf-guide.pdf
%
% Author:       Paul C. Anagnostopoulos
%               Windfall Software
%               978 371-2316
%               paul@windfall.com
%
% Created:      15 February 2005
%
%-----------------------------------------------------------------------------


\documentclass[preprint]{sigplanconf}

% The following \documentclass options may be useful:

% preprint      Remove this option only once the paper is in final form.
% 10pt          To set in 10-point type instead of 9-point.
% 11pt          To set in 11-point type instead of 9-point.
% numbers       To obtain numeric citation style instead of author/year.

\usepackage{amsmath}

\newcommand{\cL}{{\cal L}}

\begin{document}

\special{papersize=8.5in,11in}
\setlength{\pdfpageheight}{\paperheight}
\setlength{\pdfpagewidth}{\paperwidth}

%\conferenceinfo{CONF 'yy}{Month d--d, 20yy, City, ST, Country}
%\copyrightyear{20yy}
%\copyrightdata{978-1-nnnn-nnnn-n/yy/mm}
%\copyrightdoi{nnnnnnn.nnnnnnn}

% Uncomment the publication rights you want to use.
%\publicationrights{transferred}
%\publicationrights{licensed}     % this is the default
%\publicationrights{author-pays}

\titlebanner{CSC 395: Modern Programming Principles}        % These are ignored unless
\preprintfooter{Final project for CSC 395}   % 'preprint' option specified.

\title{Solving NP-Hard Problems with Neural Networks}
\subtitle{Taking the ``Artificial'' out of Artificial Intelligence}

\authorinfo{Zebediah Figura}
           {Grinnell College}
           {figuraze@grinnell.edu}

\maketitle

\begin{abstract}


\end{abstract}

\category{CR-number}{subcategory}{third-level}

\keywords
keyword1, keyword2

\section{Introduction}

The concept of neural networks, while dating back to the 1940s, has gained popularity and interest in recent years, partly due to breakthroughs in fast GPU-based implementation and the invention of recurrent neural nets. In our paper we implement a neural net in Haskell, a strongly-typed functional language, with the intent of demonstrating how a neural network can solve NP-hard problems. In order to best demonstrate the process and successfulness of a neural network, we present as the problem to be solved the well-known video game \textit{Snake}. Solving a problem such as this allows us to observe elements of the network's heuristic, as well as show the speed and manner in which it learns to solve the problem.

\appendix
\section{Appendix Title}

This is the text of the appendix, if you need one.

\acks

Acknowledgments, if needed.

% We recommend abbrvnat bibliography style.

\bibliographystyle{abbrvnat}

% The bibliography should be embedded for final submission.

\begin{thebibliography}{}
\softraggedright

\bibitem[Smith et~al.(2009)Smith, Jones]{smith02}
P. Q. Smith, and X. Y. Jones. ...reference text...

\end{thebibliography}


\end{document}
